%%=============================================================================
%% Inleiding
%%=============================================================================

\chapter{Inleiding}
\label{ch:inleiding}



\section{Probleemstelling}
\label{sec:probleemstelling en context}
Elk bedrijf doet veel moeite om klanten te werven, maar verliest de groep uit het oog die reeds klant is, Nochtans is customer-binding, customer retention en churn prevention even belangrijk. Maar ook dit vergt heel wat tijd en energie en wordt dikwijls pas in een laatste fase bekeken in het klanten proces. Machine Learning kan deze taak eventueel geautomatiseerd uitvoeren. Hierbij doet men dus een onderzoek op welke manier en met welke data we via Machine Learning geautomatiseerd kunnen aflijden wie op een bepaald moment een firma zal verlaten of bepaalde service zal opzeggen.

\section{Onderzoeksvraag}
\label{sec:Doelstelling en onderzoeksvragen}

Eerst en vooral gaan we informatie opzoeken over belangrijke termen in deze bachelorproef zoals Churn, Churn prevention, Klantenbinding, Machine Learning, Models, Deep Learning, Datasets. Aan de hand van deze informatie kan men dan kijken hoe men Machine Learning kan gebruiken voor churn of klantenbinding. Om uiteindelijk resultaat te krijgen zullen er eerst en vooral een aantal onderzoeksvragen opgelost worden:


\begin{description}
	\item [Zijn er andere technologieën die Machine Learning reeds toepassen voor churn of klantenbinding?] In deze fase gaan we op zoek gaan naar technologieën die reeds gebruik maken van Machine Learning voor churn of klantenbinding. 
	\item [Welke frameworks, visuele tools en scripting tools heb ik nodig om zelf zo een oplossing te bouwen?] In deze fase gaat men eerst kijken welke soort tools er te vinden zijn, en hoe we deze kunnen onderscheiden. Vervolgens gaat men bekijken welke soort tools het beste van toepassing zouden zijn voor ons onderzoek. Om hierna dan te zoeken naar tools die onder de gekozen soort vallen. En vervolgens een paar van deze te gaan vergelijken .
	\item [Welke data set gaan we gebruiken en hoe verkrijg ik deze data ?] Hier is het de bedoeling dat we eerst weten welke parameters belangrijk zijn voor het voorspellen van een churn, om vervolgens dan een data set zelf te maken of een data set op te zoeken op het web die parameters bevat die het meest aanleunen aan onze gekozen parameters.
	\item [Welk Model dien ik hiervoor op te zetten]Hiervoor gaan we kijken naar de algoritmes die in aanmerking komen voor het model. Vervolgens besluiten welk algoritme we gaan gebruiken.
	\item [Kies ik beter een scripting tool zoals python?]Men gaat onderzoeken waarom in het geval van een model maken, scripting tools beter zijn.
\end{description}

