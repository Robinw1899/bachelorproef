%%=============================================================================
%% Proof of Concept
%%=============================================================================

\chapter{Proof of Concept}
\label{ch:Proof of concept}


\section{specificaties van mijn systeem}

\section{installatie van TenserFlow}
Voor men begon met de proof of concept moet men eerst TenserFlow installeren. Op de site van TenserFlow staat er een handleiding hoe men TenserFlow moet installeren, en wat voor versie van TenserFlow men wilt installeren. Men kan kiezen tussen 2 versies:
TenserFlow met enkel CPU support Dit is voor de systemen die geen NVIDIA GPU hebben. Dit is een versie die makkelijker is om te installeren en men raadt ons aan om toch deze versie te installeren ook al hebben we een NVIDIA GPU
TenserFlow met GPU support. De TenserFlow programma’s werken sneller op hun GPU dan op een CPU, daarvoor als het systeem een NVIDIA GPU heeft met de nodige voorwaarden die onderaan ook vermeld staan en als men performance-critical applications moeten draaien, dan moeten we deze versie toch installeren.

•	CUDA Toolkit 9.0, je moet ook de nodige Cuba pathnames in de environment variabele toevoegen\newline
•	NVIDIA drivers die gelinkt zijn aan de CUDE Toolkit 9.0\newline
•	cuDNN v7.0 

