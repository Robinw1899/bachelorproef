%%=============================================================================
%% Samenvatting
%%=============================================================================

% TODO: De "abstract" of samenvatting is een kernachtige (~ 1 blz. voor een
% thesis) synthese van het document.
%
% Deze aspecten moeten zeker aan bod komen:
% - Context: waarom is dit werk belangrijk?
% - Nood: waarom moest dit onderzocht worden?
% - Taak: wat heb je precies gedaan?
% - Object: wat staat in dit document geschreven?
% - Resultaat: wat was het resultaat?
% - Conclusie: wat is/zijn de belangrijkste conclusie(s)?
% - Perspectief: blijven er nog vragen open die in de toekomst nog kunnen
%    onderzocht worden? Wat is een mogelijk vervolg voor jouw onderzoek?
%
% LET OP! Een samenvatting is GEEN voorwoord!

%%---------- Nederlandse samenvatting -----------------------------------------
%
% TODO: Als je je bachelorproef in het Engels schrijft, moet je eerst een
% Nederlandse samenvatting invoegen. Haal daarvoor onderstaande code uit
% commentaar.
% Wie zijn bachelorproef in het Nederlands schrijft, kan dit negeren, de inhoud
% wordt niet in het document ingevoegd.

\IfLanguageName{english}{%
\selectlanguage{dutch}
\chapter*{Samenvatting}
Elk bedrijf doet veel moeite om klanten te werven, maar verliest de groep uit het oog die reeds klant is, Nochtans is customer-binding, customer retention en churn prevention even belangrijk. Maar ook dit vergt heel wat tijd en energie en wordt dikwijls pas in een laatste fase bekeken in het klanten proces. Machine Learning kan deze taak eventueel geautomatiseerd uitvoeren. Hierbij doet men dus een onderzoek op welke manier en met welke data we via Machine Learning geautomatiseerd kunnen afleiden wie op een bepaalde moment een firma zal verlaten of bepaalde service zal opzeggen.

In deze Bachelorproef heeft men op basis van sites en algemeen verkregen documentatie bekeken wat de mogelijkheden zijn om dit onderzoek te volbrengen
Hierbij is men onder andere tot de conclusie gekomen dat klanten meer op emotioneel niveau kijken of ze bij een bepaalde firma blijven. Maar dit is niet van toepassing voor een aankoop op een site. Dus heeft men zich dan gebaseerd op het rationeel niveau. 
Na de literatuurstudie begon men de onderzoeksvragen elk apart te beantwoorden.
Men heeft gezien dat er nog andere technologieën zijn die Machine Learning gebruiken voor churn prevention. Namelijk: Dataiku, Databricks en de IBM Watson Analytics. Men heeft de verschillende soorten tools onderzocht en bekeken welke tools men het beste zou gebruiken voor zelf een Machine learning model te maken. De conclusie was dat we een library, API, local tool gaan gebruiken. Hieronder hebben we 2 tools gevonden op het internet :DeepLearningforj en TenserFlow. Vervolgens hebben we gekozen voor TenserFlow. Vanwege xyz. Men heeft onderzocht wat voor dataset men nodig had om het model op te trainen. Gelukkig was er een model dat men kon vinden en konden gebruiken. De Telco-Customer-Churn dataset. 

\selectlanguage{english}
}{}

%%---------- Samenvatting -----------------------------------------------------
% De samenvatting in de hoofdtaal van het document

\chapter*{\IfLanguageName{dutch}{Samenvatting}{Abstract}}

\lipsum[1-4]
